\section{Lösungen Herleiten}
\subsection{Umverteilung des Gewichtes}
Da der Lidar Sensor eine Neigung um \(7^\circ\) nach vorne hat (schwerer, ungestützter Vorbau), muss das Gewicht neu verteilt werden. Hierfür werden drei Möglichkeiten herausgearbeitet und in Tabelle~\ref{tab:nutzwert_gewicht} verglichen.

\subsubsection{Hochkante Einlage in den Mittelteil}
\paragraph{Art der Änderung}
\begin{itemize}
  \item Der Vorderbau wird mechanisch begradigt: die Bereiche um die Schraubpunkte werden abgefeilt, bis eine rechteckige, sauber einpassbare Form entsteht.
  \item Der Vorderbau wird anschließend hochkant in den Mittelteil integriert.
  \item Fixierung über einen Winkel hinter der Servolenkung, damit die Einheit verwindungssteif sitzt und Kräfte definiert eingeleitet werden.
  \item Das Batteriefach wird hochkant ausgerichtet und in Richtung Motor verschoben, um den Bauraum besser zu nutzen und den Schwerpunkt sinnvoll zu verlagern.
  \item Leitungsführung und Befestigungspunkte müssten dabei neu geplant werden, weil sich Einbauwinkel und Zugänglichkeit ändern.
\end{itemize}

\paragraph{Vorteile}
\begin{itemize}
  \item Sehr platzsparende Integration, kaum toter Bauraum.
  \item Wendekreis bleibt im Normalfall unverändert, da Radstand und Lenkeinschlag nicht zwangsläufig angepasst werden müssen.
  \item Konstruktion wirkt insgesamt kompakter und aufgeräumter, was spätere Erweiterungen erleichtern kann.
\end{itemize}

\paragraph{Nachteile}
\begin{itemize}
  \item Wartbarkeit kann schlechter werden: Akkuwechsel, Schrauben nachziehen, Komponenten tauschen wird je nach Einbauposition fummeliger.
  \item Höherer mechanischer Anpassungsaufwand (Feilen, Ausrichten, Winkel fertigen), dadurch mehr Risiko für Passungenauigkeiten.
  \item Kabelmanagement wird kritischer, weil weniger Platz für saubere Radien und Zugentlastung bleibt.
\end{itemize}

\subsubsection{Frankenstein Umbau}
\paragraph{Art der Änderung}
\begin{itemize}
  \item Vorderbau wird vollständig abgeschraubt.
  \item Mittelteil des Chassis wird durchgesägt, um eine neue Montageposition zu schaffen.
  \item Der Vorderbau wird in die Mitte versetzt und dort wieder verschraubt.
  \item Je nach Schnittstelle müssten Verstärkungen ergänzt werden (Platten, Winkel, zusätzliche Verschraubungen), damit die Struktur unter Last nicht nachgibt.
\end{itemize}

\paragraph{Vorteile}
\begin{itemize}
  \item Konzeptuell einfach, weil umsetzen und festschrauben als Grundidee klar ist.
  \item Kann schnell einen deutlichen Effekt auf Schwerpunkt und Platzverteilung bringen, ohne viele Detailarbeiten an einzelnen Bauteilen.
\end{itemize}

\paragraph{Nachteile}
\begin{itemize}
  \item Wendekreis kann schlechter werden, weil sich Geometrie und Gewichtsverteilung ungünstig verändern können.
  \item Stabilitätsrisiko: Durchsägen schwächt das Chassis, dadurch Verwindung, Risse, lockere Verschraubungen möglich.
  \item Oft mehr Nacharbeit als gedacht, weil nach dem groben Umbau Verstärkungen und Ausrichtung notwendig werden.
\end{itemize}

\subsubsection{Einkaufsrad Lösung}
\paragraph{Art der Änderung}
\begin{itemize}
  \item Ein Stützrad (Einkaufsrad) wird unter dem Vorderbau montiert.
  \item Ziel ist, dass das Rad einen Teil der Frontlast trägt und das Fahrzeug vorne mechanisch stabilisiert.
  \item Befestigung über eine einfache Halterung oder Montageplatte, ohne größere Eingriffe in den Aufbau.
\end{itemize}

\paragraph{Vorteile}
\begin{itemize}
  \item Sehr schnell und kostengünstig umsetzbar.
  \item Minimal invasiv: bestehender Aufbau bleibt weitgehend erhalten, Rückbau ist einfach.
  \item Sofortiger Effekt auf Lastverteilung, ohne Chassis oder Hauptbaugruppen umzubauen.
\end{itemize}

\paragraph{Nachteile}
\begin{itemize}
  \item Optisch eher Bastellösung, wirkt weniger integriert.
  \item Fahrverhalten kann schlechter werden: Nachlaufen, Flattern, zusätzliche Reibung, eingeschränkte Bodenfreiheit.
  \item Auf unebenem Untergrund oder bei Kanten kann das Stützrad stören oder hängen bleiben.
\end{itemize}

\subsubsection{Nutzwertanalyse zur Gewichtsumverteilung}
\begin{table}[htbp]
\centering
\includegraphics[width=\textwidth]{tables/Nutzweranalyse_Gewichtsverteilung.png}
\caption{Nutzwertanalyse zur Gewichtsverteilung}
\label{tab:nutzwert_gewicht}
\end{table}

Aus der Nutzwertanalyse in Tabelle~\ref{tab:nutzwert_gewicht} ergibt sich, dass die Hochkante Einlage im Mittelteil die beste Lösung zur Umverteilung des Gewichtes darstellt. Die Lösung bietet eine gute Balance aus Umsetzbarkeit, Fahrverhalten und Wartbarkeit, während sie gleichzeitig den Schwerpunkt effektiv verlagert. Obwohl der mechanische Anpassungsaufwand höher ist, überwiegen die Vorteile in Bezug auf Platznutzung und Fahrzeugperformance. Daher wird diese Lösung für die weitere Umsetzung empfohlen. 
