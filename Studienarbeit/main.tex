\documentclass[a4paper,12pt]{article}

% -----------------------------------------
% Pakete und Einstellungen
% -----------------------------------------
\usepackage[a4paper, left=2.5cm, right=2.5cm, top=1cm, bottom=2.5cm, includehead]{geometry} 
\usepackage[ngerman]{babel}
\usepackage{fontspec}
\setmainfont{Arial}
\usepackage{csquotes}
\usepackage{graphicx}
\usepackage{float}
\usepackage{hyperref}
\usepackage{listings}
\usepackage{color}
\usepackage[
    backend=biber,
    style=numeric,
    sorting=nyt
]{biblatex}
\makeatletter
\let\blx@warn@emptybib\relax
\makeatother
\usepackage{setspace}
\usepackage{graphicx}
\graphicspath{
  {images/}
  {tables/}
}
\usepackage{caption}
\usepackage[acronym,nonumberlist]{glossaries}
\makeglossaries
\usepackage{booktabs}
\usepackage{multirow}
\usepackage[table]{xcolor}
\usepackage{array}
\usepackage{adjustbox}  % max width=\textwidth
\addbibresource{literatur/T2000.bib}


\definecolor{lightgray}{rgb}{0.95,0.95,0.95}
\lstset{
    backgroundcolor=\color{lightgray},
    basicstyle=\ttfamily\small,
    breaklines=true,
    frame=single
}
\setstretch{1.5}                % Zeilenabstand: 1.5 (Standard: 1.2)
\begin{document}
\pagenumbering{gobble} % unterdrückt Seitenzahlen komplett

\begin{titlepage}
    \begin{flushright}
        \includegraphics[width=0.28\textwidth]{Images/DHBWlogo.png}
    \end{flushright}

    \vspace*{1.5cm}

    \begin{center}
        {\large Weiterentwicklung eines selbstfahrenden Fahrzeugs mit Lidar und anderen Sensoren}

        \vspace{2cm}

        {\bfseries \large STUDIENARBEIT}
    \end{center}

    \vspace{1.5cm}

    \begin{center}
        des Studienganges {Informatik} \\
        an der Dualen Hochschule Baden-Württemberg Ravensburg
    \end{center}

    \vspace{1cm}

    \begin{center}
        von \\
        {Elena Schwarzbach, 3830156, TIK23

        Sonia Sinaci, Matrikelnummer, TIK23

        Scott Jonathan Hebach, Matrikelnummer, TIT23

        Marius Maurer, 9339665, TIT23}
    \end{center}

    \vspace{1.5cm}

    \begin{center}
        {Abgabedatum}
    \end{center}

    \vspace{1.5cm}

    \noindent
    \begin{tabular}{@{}ll}
        Bearbeitungszeitraum:           & {x Wochen}                                 \\
        Gutachter der Dualen Hochschule:& {Titel Vorname Nachname}                    \\
    \end{tabular}
\end{titlepage}
\thispagestyle{empty}

\begin{flushleft}
	\Huge\bf Eidesstattliche Erklärung\\
	\vspace*{3cm}
	\normalsize\rm
	gemäß Ziffer 1.1.13 der Anlage 1 zu §§ 3, 4 und 5 der Studien- und Prüfungsordnung für die Bachelorstudiengänge im Studienbereich Technik der Dualen Hochschule Baden-Württem-berg vom 29.09.2017.
\vspace{1cm}

Ich versichere hiermit, dass ich meine Bachelorarbeit (bzw. Projektarbeit oder Studienarbeit bzw. Hausarbeit) mit dem Thema:
\vspace{1cm}

{\large \bf Weiterentwicklung eines selbstfahrenden Fahrzeugs mit Lidar und anderen Sensoren}

\vspace{1cm}
selbstständig verfasst und keine anderen als die angegebenen Quellen und Hilfsmittel benutzt habe. Ich versichere zudem, dass die eingereichte elektronische Fassung mit der gedruckten Fassung übereinstimmt.\\
	\vfill
\noindent
  
\noindent
\noindent
\begin{tabular}{p{7cm} p{7cm}}
Friedrichshafen, TT.MM.JJJJ & \\[1cm]
\hrulefill & \hrulefill \\
Elena Schwarzbach  & Sonia Sinaci \\
[1.5cm]
\hrulefill & \hrulefill \\
Scott Jonathan Hebach & Marius Maurer \\
\end{tabular}


\end{flushleft}
\thispagestyle{empty}

\begin{flushleft}
	\Huge\bf Genderhinweis\\
	\vspace*{2cm}
	\normalsize\rm
In der vorliegenden Arbeit wird aus Gründen der sprachlichen Ökonomie und zur Sicherung eines einheitlichen Schriftbildes auf eine durchgängige geschlechts-spezifische Differenzierung verzichtet. Sämtliche Personen-bezeichnungen im generischen Maskulinum verstehen sich als geschlechtsneutral und schließen alle Geschlechter gleichermaßen ein.

Diese Entscheidung basiert rein auf redaktionellen Überlegungen und spiegelt keine Geringschätzung gegenüber weiblichen oder nicht-binären Personen wider. Die Autoren sind sich der Wirkung von Sprache auf die Wahrnehmung bewusst und verfolgen mit dieser Wahl das Ziel, die Komplexität des Textes zugunsten der inhaltlichen Klarheit zu reduzieren.
\end{flushleft}
\thispagestyle{empty}

\cleardoublepage
\pagenumbering{Roman}
\tableofcontents
\newpage

\newacronym{ki}{KI}{Künstliche Intelligenz}
\newacronym{api}{API}{Application Programming Interface}

\addcontentsline{toc}{section}{Abkürzungsverzeichnis}
\glsaddall[types={\acronymtype}]
\printglossary[type=\acronymtype,title=Abkürzungsverzeichnis]
\newpage

\addcontentsline{toc}{section}{Abbildungsverzeichnis}
\listoffigures
\newpage

\addcontentsline{toc}{section}{Tabellenverzeichnis}
\listoftables
\newpage


% usw.
\pagenumbering{arabic}
\section{Einleitung}
In diesem Kapitel wird aufgezeigt warum sich für dieses Projekt entschieden hat und was gefordert ist und welche Gegebenheiten vorliegend sind.
\subsection{Motivation}
Die Entscheidung für dieses Projekt beruht auf dem Wunsch, ein Thema zu wählen, das einen klaren Bezug zur Praxis hat und gleichzeitig eine große Bedeutung für zukünftige Anwendungen besitzt. Ein autonomes Fahrzeug ist ein Paradebeispiel für Systeme, die eigenständig reale Umgebungen wahrnehmen, darauf reagieren und somit Aufgaben übernehmen, die größtenteils menschliche Aufmerksamkeit erforderten. Die Verbindung von direkter Anwendbarkeit und großem Entwicklungs-potenzial war der Grund, warum das Projekt ausgewählt wurde. Es liefert ein konkretes Resultat, welches nachvollziehbar demonstrierbar ist und im Laufe des Projekts sukzessive optimiert werden kann.
\\\\
Die Zukunftsrelevanz wird durch die zunehmende Verbreitung von autonomen und assistierenden Systemen in vielen Anwendungsfeldern deutlich. Neben der herkömmlichen Mobilität werden solche Lösungen in der Logistik, in industriellen Umgebungen, in kommunalen Anwendungen sowie in Assistenzszenarien immer wichtiger, besonders dort, wo es darum geht, Prozesse effizienter, sicherer oder zuverlässiger zu gestalten. Nicht nur als einzelnes Produkt stehen autonome Fahrzeuge da, sie sind auch ein Zeichen für eine größere Evolution hin zu selbstständig agierenden, datengestützten Systemen, die in der Lage sind, in dynamischen Situationen Entscheidungen zu treffen. Das Projekt ist ein wichtiger Beitrag, weil es praxisnah zentrale Anforderungen der Zukunft untersucht und die Erkenntnisse systematisch dokumentiert.
\\\\
Insgesamt verbindet das Projekt eine motivierende Zielsetzung mit einer klaren Perspektive: Es schafft einen Projektgegenstand, der aktuelle Entwicklungen aufnimmt und zugleich die Anwendungen in den Blick nimmt, die in den kommenden Jahren zunehmend relevant sein werden.


\subsection{Zielsetzung}
Ein Fahrzeug, welches von Studierenden der Elektrotechnik und Informatik entworfen und programmiert wurde, ist bereits vorhanden. Im Projekt wird dieses Fahrzeug weiterentwickelt, um autonomes Fahren zu ermöglichen. Der aktuelle Stand beinhaltet eine vorhandene Sensorik, insbesondere einen LiDAR-Sensor sowie mehrere Ultraschallsensoren, die zur Umfelderfassung eingesetzt werden können.
\\\\
Die Zielsetzung der Weiterentwicklung umfasst die Bedienung über ein Smartphone, um das Fahrzeug manuell steuern zu können, sowie die Ergänzung eines Selbstfahrmodus. Im Selbstfahrmodus soll das Fahrzeug autonom innerhalb einer festgelegten Umgebung navigieren und Hindernissen ausweichen können. So ist nicht nur die Bewegung an sich wichtig, sondern vor allem ein verlässliches Ausweichverhalten, welches den autonomen Betrieb maßgeblich unterstützt.
\\\\
Die Herausforderung besteht darin, ein bereits aufgebautes System so zu erweitern, dass es sowohl einen robusten Fernsteuerbetrieb als auch einen autonomen Betriebsmodus bietet, indem die vorhandene Sensorik dafür gezielt genutzt wird.
\subsection{Problemstellung}
Das Auto hat in der Elektronik, Mechanik und Software Probleme, welche im Verlauf der Weiterentwicklung behoben werden müssen.
\\
Elektronik:
\begin{itemize}
    \item kein Verdrahtungsplan
    \item abgebrochen Kabel
    \item Kabel die ins nichts führen
    \item kurze Akkulaufzeit
\end{itemize}
Mechanik:
\begin{itemize}
    \item Frontalneigung um 7° wegen zu schwerem Vorbau
    \item kompaktes und schlecht wartbares Design
    \item fehlende/kaputte Halterungen für Sensoren
\end{itemize}
Software:
\begin{itemize}
    \item benötigt große Speicherkapazität
    \item hohe Arbeitsspeichernutzung
    \item unzureichende Dokumentation \gls{ki}
\end{itemize}
 \newpage
\section{Methodiken}
In diesem Kapitel werden Methodiken zur Herleitung oder Abwägung einer Lösung erklärt. Dies ist förderlich, um im Verlauf einen möglichst effizienten Lösungsweg zu schaffen.

\subsection{Nutzwertanalyse}

\definecolor{nwGray}{RGB}{191,191,191}
\definecolor{nwGreen}{RGB}{99,190,123}
\definecolor{nwRed}{RGB}{248,105,107}
\definecolor{nwYellow}{RGB}{255,235,132}

\begin{table}[htbp]
\centering
\small
\setlength{\tabcolsep}{4pt}
\renewcommand{\arraystretch}{1.25}

\begin{adjustbox}{max width=500pt}
\begin{tabular}{|l|c|c|c|c|c|c|c|c|}
\hline
 & \multirow{2}{*}{\rotatebox{90}{Wertebereich}} &
 \multirow{2}{*}{\rotatebox{90}{Gewichtung}} &
 \multicolumn{2}{c|}{Option 1} &
 \multicolumn{2}{c|}{Option 2} &
 \multicolumn{2}{c|}{Option 3} \\
\cline{4-9}
 &  &  &
 \rotatebox{90}{Bewertung} & \cellcolor{nwGray}\rotatebox{90}{Nutzwert} &
 \rotatebox{90}{Bewertung} & \cellcolor{nwGray}\rotatebox{90}{Nutzwert} &
 \rotatebox{90}{Bewertung} & \cellcolor{nwGray}\rotatebox{90}{Nutzwert} \\
\hline

Kriterium 1 & 0-5 & 10 & 1 & \cellcolor{nwRed}{10}  & 3 & \cellcolor{nwGreen}{30} & 3 & \cellcolor{nwGreen}{30} \\
\hline
Kriterium 2 & 0-5 & 30 & 2 & \cellcolor{nwRed}{60}  & 5 & \cellcolor{nwGreen}{150} & 2 & \cellcolor{nwRed}{60} \\
\hline
Kriterium 3 & 0-5 & 25 & 4 & \cellcolor{nwGreen}{100} & 1 & \cellcolor{nwRed}{25}  & 4 & \cellcolor{nwGreen}{100} \\
\hline
Kriterium 4 & 0-5 & 25 & 3 & \cellcolor{nwGreen}{75}  & 2 & \cellcolor{nwRed}{50}  & 3 & \cellcolor{nwGreen}{75} \\
\hline
Kriterium 5 & 0-5 & 15 & 3 & \cellcolor{nwGreen}{45}  & 2 & \cellcolor{nwRed}{30}  & 2 & \cellcolor{nwRed}{30} \\
\hline

\multicolumn{4}{|c|}{Nutzwertsumme} &
\cellcolor{nwYellow}{290} &
\multicolumn{1}{c|}{} & \cellcolor{nwRed}{285} &
\multicolumn{1}{c|}{} & \cellcolor{nwGreen}{295} \\
\hline

\multicolumn{4}{|c|}{Rangplatz} &
\cellcolor{nwYellow}{2} &
\multicolumn{1}{c|}{} & \cellcolor{nwRed}{3} &
\multicolumn{1}{c|}{} & \cellcolor{nwGreen}{1} \\
\hline
\end{tabular}
\end{adjustbox}

\caption{Beispiel Nutzwertanalyse}
\label{tab:nutzwertanalyse}
\end{table}
Eine Nutzwertanalyse (siehe Tabelle 1 [1]) ist eine Möglichkeit, aus vielen Optionen, die bestimmte Kriterien erfüllen müssen, die beste Option zu finden. Sie ist in verschiedene Spalten aufgeteilt. Die linke Spalte enthält die Kriterien, welche erfüllt sein müssen, damit das Endergebnis gut ist. Die folgende Spalte enthält den Wertebereich der verteilt werden kann. Spalte drei könnte optional eine Gewichtung der Kriterien festlegen, dadurch können genauere Ergebnisse erzielt werden. Diese Nutzwertanalyse wäre in dem Fall eine gewichtete Nutzwertanalyse, ohne Gewichtung wäre es nur eine einfache Nutzwertanalyse. Anschließend an die ersten Spalten kommen nun die Optionsspalten. Diese werden Spalte für Spalte durchgearbeitet und Punkte vergeben. Falls hinter Punkten eine Gewisse Bedeutung liegt, ist die Erstellung einer Legende notwendig. Für das Ergebnis werden nun die Eintragungen mit der Gewichtung der Zeile multipliziert und dann die Optionsspalte addiert. Anhand des Ergebnisses kann dann auswertet werden, welche Option am besten die Kriterien erfüllt. Die farbliche Abhebungen wird genutzt, dass schneller erkenntlich ist welche Option ein Kriterium am besten erfüllt.\newpage
\section{Grundlagen}
\subsection{Überblick Autonomes Fahren} 
(bei Modellautos)
\subsection{LIDASensorik} 
in autonomen Systemen: Prinzipien, etablierte Lösungen

\subsection{Relevante Algorithmen und Architekturkonzepte}
(mit Blick auf wissenschaftliche Literatur)

•	Überblick über SLAM, Hinderniserkennung, Navigation
\newpage
\section{Ist-Zustand und wissenschaftliche Bewertung der Sesorik}
•	Anforderungen an die Sensorik im Kontext des Projektziels (nur so tief wie nötig)\\
•	Wissenschaftliche Diskussion:

o	„Braucht man (diese) Sensoren wirklich?“ – Vergleich verschiedener Sensortechnologien (LIDAR, Kamera, Ultraschall)

o	Vor- und Nachteile im konkreten Anwendungsfall (Modellfahrzeug, Innenraum, Umgebungseinflüsse)

o	Qualitäts- versus Mengenfrage: Wenige hochwertige vs. viele günstige Sensoren

o	Einflüsse defekter bzw. nicht verfügbarer Hardware auf die Projektentscheidungen\\
•	Fundierte, nachvollziehbare Begründung der gewählten Sensorik (ggf. Änderung des Sensor-Setups)


LIDAR Sensor: 
In lidar_data.py wird der typische LIDAR‑Pipeline‑Ablauf umgesetzt. 
Ein JSON‑Scan wird geladen, aus angle_min, angle_max und angle_increment werden die Messwinkel berechnet, und die zugehörigen Distanzen aus ranges werden zusammengeführt. Anschließend werden die Daten in einem DataFrame strukturiert und als Polarplot visualisiert, wodurch die räumliche Verteilung der Messpunkte um den Sensor sichtbar wird.
Die Scan‑Dateien lidar_scan_0.json bis lidar_scan_8.json sind konsistent aufgebaut: Jeder Scan umfasst 1147 Messpunkte und deckt den Vollkreis von etwa −π bis +π ab, bei einer Winkelauflösung von ca. 0,00548 rad (~0,314°). Die Distanzwerte liegen typischerweise zwischen ca. 0,47 m und 10,4 m, der Mittelwert bewegt sich stabil im Bereich von 3,07–3,11 m. Ein signifikanter Anteil der Werte ist nicht‑finite (ca. 268–278 pro Scan), was auf Messausfälle oder Bereiche außerhalb der Sensorreichweite hindeutet. Dies ist bei LIDAR‑Messungen üblich und sollte in der Auswertung gefiltert oder als „kein Treffer“ behandelt werden. Insgesamt deuten die Daten auf eine gleichmäßige und zuverlässige Erfassung mit konsistenter Sensorauflösung und stabiler Reichweite hin.

Im aktuellen Projekt werden die LIDAR‑Daten an zwei Stellen genutzt: 
o lidar_data.py lädt direkt lidar_scan_5.json (und kann beliebige lidar_scan_*.json verarbeiten) für DataFrame‑Analyse und Polarplot. 
o lidar_data.dart hingegen lädt nicht die lidar_scan_*.json, sondern das occupancy_grid_example.json und erzeugt daraus ein Occupancy‑Grid für die Flutter‑UI, die Struktur des Grids ist in occupancy_grid.dart definiert.

Eine operative Nutzung realer LIDAR‑Hardware ist im vorliegenden Projektstand nicht implementiert. Die vorhandenen LIDAR‑bezogenen Dateien dienen ausschließlich der Offline‑Analyse bzw. Visualisierung: 
o In lidar_data.py werden statische JSON‑Scans aus dem Asset‑Ordner geladen und geplottet, während die Flutter‑App ein Beispiel‑Occupancy‑Grid aus den Assets darstellt. 
o Es fehlen Treiber‑ oder Kommunikationsschnittstellen (z. B. ROS‑Topics, Serial, UDP/TCP) zur Anbindung eines physischen LIDAR‑Sensors. 
Somit ist der LIDAR im aktuellen Systemzustand nicht funktional integriert und die Daten stammen aus extern bereitgestellten, vorab aufgezeichneten Datensätzen.



\newpage
\section{Konzept und Anforderungsdefinition}
•	Anforderungen an das Zielsystem (abgeleitet aus Aufgabenstellung)\\
•	Detaillierte Auflistung und fachliche Begründung der geplanten Weiterentwicklungen\\
•	Ableitung der Architektur auf Basis der Sensorik-Entscheidungen
\newpage
\section{Implementierunng}
\subsection{Methodisches Vorgehen} 
(z.B. agile Entwicklung, Prototyping)\\
•	Auswahl/Begründung der eingesetzten Algorithmen/Architekturen\\
•	Schwerpunktsetzung (z.B. Pfadplanung, Sensorfusion, Datenverarbeitung) mit Verweis auf wissenschaftliche Relevanz\\
•	Detaillierte Beschreibung der wichtigsten Entwicklungsaspekte (z.B. SLAM-Implementierung, Optimierung von Sensordaten, Integration in App/UI)\\
•	Umgang mit Hardwareproblemen: Auswirkungen auf das Softwaredesign, ggf. Umgehungslösungen
\newpage
\section{Evaluation und Validierung}
\subsection{Definition von Testkriterien und -szenarien}
\subsection{Messmethoden und Metriken}
Überblick zu Messmethoden und -metriken (z.B. Genauigkeit der Kartierung, Latenz der Sensorik, Zuverlässigkeit der Abstandswarnung)
\subsection{Testergebnisse}
Darstellung, Analyse, Interpretation
Diskussion von Problemen, Limitationen und offenen Punkten
\newpage
\section{Kritische Reflexion}
•	Bewertung der gewählten Ansätze im Vergleich zu Alternativen\\
•	Wissenschaftliche Einordnung des eigenen Beitrags\\
•	Lessons Learned: Was ließ sich nicht wie geplant umsetzen? Welche (wissenschaftlichen) Fragen bleiben offen?
\newpage
\section{Fazit und Ausblick}
•	Zusammenfassung der zentralen Erkenntnisse und Beiträge\\
•	Potenziale für weitere wissenschaftliche Arbeiten oder Produktentwicklungen

\newpage

\cleardoublepage
\pagenumbering{Roman}
\setcounter{page}{5}

\cite{dummyhidden}
\addcontentsline{toc}{section}{Literaturverzeichnis}
\printbibliography[title={Literaturverzeichnis}]

\par\noindent\textit{Es wurden keine Literaturverweise im Dokument gesetzt.}



\appendix
\end{document}
